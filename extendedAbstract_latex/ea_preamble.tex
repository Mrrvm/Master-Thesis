%%%%%%%%%%%%%%%%%%%%%%%%%%%%%%%%%%%%%%%%%%%%%%%%%%%%%%%%%%%%%%%%%%%%%%
% Document preamble
%%%%%%%%%%%%%%%%%%%%%%%%%%%%%%%%%%%%%%%%%%%%%%%%%%%%%%%%%%%%%%%%%%%%%%

%% Builds upon the graphics  package, providing a key-value interface
%% for optional arguments to the \includegraphics command that go far
%% beyone what the graphics package offers.
%% http://www.ctan.org/tex-archive/help/Catalogue/entries/graphicx.html
%% if you use PostScript figures in your article
%% use the graphics package for simple commands
%% \usepackage{graphics}
%% or use the graphicx package for more complicated commands
%% \usepackage{graphicx}
%% or use the epsfig package if you prefer to use the old commands
%% \usepackage{epsfig}
\usepackage{graphicx} % Enhanced LaTeX Graphics

% Multiple figures
%\usepackage{subfigure} % subcaptions for subfigures
%\usepackage{subfigmat} % matrices of similar subfigures

% Declaring new column types
% 'dcolumn' package defines D to be a column specifier with
% three arguments: D{<sep.tex>}{<sep.dvi>}{<decimal places>}
%                  D{<sep.tex>}{<sep.dvi>}{<left digit places>.<right digit places>}
\usepackage{dcolumn}           % decimal-aligned tabular math columns
% d takes a single argument specifying the number of decimal places, e.g., d{2}
% or the number of digits to the left and right of the seperator, e.g., d{3.2}
\newcolumntype{.}   {D{.}{.}{-1}} % column alignedd on the point separator '.'
\newcolumntype{d}[1]{D{.}{.}{#1}} % column centered on the point separator '.'
\newcolumntype{e}   {D{E}{E}{-1}} % column centered on the exponent 'E'
\newcolumntype{E}[1]{D{E}{E}{#1}} % column centered on the exponent 'E'

%% American Mathematical Society (AMS) plain Tex macros
%%
%% The amsmath package is the principal package in the AMS-LaTeX distribution
%% http://www.ctan.org/tex-archive/help/Catalogue/entries/amsmath.html
\usepackage{amsmath}
%%
%% The amsfonts package provides extended TeX fonts
%% http://www.ctan.org/tex-archive/help/Catalogue/entries/amsfonts.html
\usepackage{amsfonts}
%% The amssymb package provides various useful mathematical symbols
\usepackage{amssymb}
%%
%% The amsthm package provides extended theorem environments
%% http://www.ctan.org/tex-archive/help/Catalogue/entries/amsthm.html
\usepackage{amsthm}

%% Improves the interface for defining floating objects such as figures and tables.
%% The package also provides the H float modifier option of the obsolete here package.
%% http://www.ctan.org/tex-archive/help/Catalogue/entries/float.html
\usepackage{float}

%% Control sectional headers. 
%% http://www.ctan.org/tex-archive/help/Catalogue/entries/sectsty.html
\usepackage{sectsty}
%%
%% Redefine the font size of the 'section' and 'subsection' headings
\newcommand{\myFontSize}{\fontsize{11}{0}\selectfont}
\newcommand{\sectionFontSize}{\fontsize{12}{0}\selectfont}
\newcommand{\subsubsectionFontSize}{\fontsize{10}{0}\selectfont}
\sectionfont{\sectionFontSize}       % 10pt, Bold face (default)
\subsectionfont{\myFontSize} % 10pt, Plain face
\subsubsectionfont{\subsubsectionFontSize} % 10pt, Plain face

%\subsubsectionfont{\rm\myFontSize} % 10pt, Plain face

%% Select alternative section titles.
%% http://www.ctan.org/tex-archive/help/Catalogue/entries/titlesec.html
\usepackage{titlesec}
%%
%% Left indent, before and after spacing
%% (The starred version kills the indentation of the paragraph following the title)
\titlespacing*{\section}{0pt}{10pt}{0pt}
\titlespacing*{\subsection}{0pt}{10pt}{0pt}
\titlespacing*{\subsubsection}{0pt}{10pt}{0pt}

%% Section numbers with trailing dots. 
%% http://www.ctan.org/tex-archive/help/Catalogue/entries/secdot.html
\usepackage{secdot}
%%
%% Also put a dot after the subsection number
\sectiondot{subsection}
%% Set a space between dot and heading text
\sectionpunct{section}{. }    % By default, \sectiondot places a \quad
\sectionpunct{subsection}{. } % after the number

% These are exact settings for a A4 page with top margin of
% 25 mm, bottom margin of 30 mm, left and right margins of 25 mm,
% printable area 242 X 160 mm.

\setlength{\topmargin}{-10.4mm}
\setlength{\headheight}{0.0mm}
\setlength{\headsep}{10.0mm}
\setlength{\textwidth}{160mm}
\setlength{\textheight}{242mm}
\setlength{\oddsidemargin}{0mm}
\setlength{\evensidemargin}{0mm}
\setlength{\marginparwidth}{0mm}
\setlength{\marginparsep}{0mm}

% New command to refer to equations as Eq.(1),Eq.(2),...
\newcommand{\eqnref}[1]{Eq.(\ref{#1})}

%\usepackage{algorithm2e}
\usepackage[linesnumbered,commentsnumbered,ruled,vlined]{algorithm2e}
\newcommand\mycommfont[1]{\footnotesize\ttfamily\textcolor{black}{#1}}
\SetCommentSty{mycommfont}

%lined,boxed
\usepackage{pgf,tikz,pgfplots}
\pgfplotsset{compat=1.15}
\usepackage{mathrsfs}
\usetikzlibrary{arrows}
\usetikzlibrary{calc}
\definecolor{rvwvcq}{rgb}{0.08235294117647059,0.396078431372549,0.7529411764705882}


\usepackage{tikz}
\usetikzlibrary{positioning}
\usetikzlibrary{arrows.meta}
\usetikzlibrary{shapes.callouts}
\usetikzlibrary{fit}
\usetikzlibrary{3d}

\usepackage{caption}
\usepackage{subcaption}
\usepackage{comment}

\usepackage{dsfont}
\usepackage{relsize}
\long\def\/*#1*/{}

\newcommand\independent{\protect\mathpalette{\protect\independenT}{\perp}}
\def\independenT#1#2{\mathrel{\rlap{$#1#2$}\mkern2mu{#1#2}}}

\newcommand{\obs}{\mathbf{X}}
\newcommand{\obsvalues}{\mathbf{x}}
\newcommand{\obsallvalues}{\Tilde{\mathbf{x}}}

\newcommand{\hidden}{\mathbf{H}}
\newcommand{\hiddenvalues}{\mathbf{h}}
\newcommand{\hiddenallvalues}{\Tilde{\mathbf{h}}}

\newcommand{\ptarget}{P_{\mathrm{target}}}
\newcommand{\ptheta}{P_\theta}

\newcommand{\etarget}{\mathbb{E}_{\mathrm{target}}}
\newcommand{\etheta}{\mathbb{E}_\theta}

\newcommand{\dataset}{\mathcal{D}}
\newcommand{\datasetsize}{D}

\newcommand{\energy}{E_{\theta}}

%\usepackage{algorithm2e}
\usepackage[linesnumbered,commentsnumbered,ruled,vlined]{algorithm2e}
%lined,boxed
\usepackage{pgf,tikz,pgfplots}
\pgfplotsset{compat=1.15}
\usepackage{mathrsfs}
\usetikzlibrary{arrows}
\usetikzlibrary{calc}
\definecolor{rvwvcq}{rgb}{0.08235294117647059,0.396078431372549,0.7529411764705882}

\usepackage{times}

\usepackage{enumitem}
\usepackage{pdflscape}

\definecolor{golden}{rgb}{1.0, 0.84, 0.0}

\usepackage[toc]{glossaries}

% Glossary Definition

\newglossaryentry{homo}
{	
	name={homogeneous coordinates}, 
	description={In computer vision, an extra dimension is added to the coordinates of a point that makes perspective projection transformations easier to compute. For a point $[x \ y]^T$, any three numbers, $[a_1 \ a_2 \ a_3]^T$ for which $\frac{a_1}{a_3} = x$ and $\frac{a_2}{a_3} = y$ are homogenous coordinates. This three new coordinates are represented by "$\sim$", e.g. $\widetilde{a} = [a_1 \ a_2 \ a_3]^T$}
}

\newglossaryentry{gauss}
{	
	name={gaussian blur}, 
	description={The Gaussian blur is an image-blurring filter that uses a Gaussian function. It produces a smoother and less noisy image.}
}

\newglossaryentry{svd}
{
	name={singular value decomposition},
	description={\acrfull{svdd} is  the  factorization  of a matrix A into  the product of three matrices $U$$\Sigma$$V^T$   where the columns of $U$ and $V$ are orthonormal and $\Sigma$ is a diagonal matrix with positive real entries}
}

\newglossaryentry{skews}
{
	name={skew symmetric matrix},
	description={A skew-symmetric matrix is a square matrix whose transpose equals its negative. These matrices can be used to represent cross products as matrix multiplications.}
}

\newglossaryentry{lag}
{
	name={langrange multipliers},
	description={Using lagrange multipliers is a strategy to find the minima or maxima of a function subject to equality constraints. This works by converting the constrained problem into a form such that the derivative test of an unconstrained problem can still be applied. The lagrangian form of a a function $f(x)$ subject to $g(x) = 0$ would be $\mathcal{L}(x, \lambda)=f(x)+\lambda g(x)$}
}


\usepackage[numbers,sort&compress]{natbib}
\usepackage{notoccite}


\usepackage[pdftex]{hyperref} % enhance documents that are to be
                              % output as HTML and PDF
\hypersetup{colorlinks,       % color text of links and anchors,
                              % eliminates borders around links
%            linkcolor=red,    % color for normal internal links
            linkcolor=black,  % color for normal internal links
            anchorcolor=black,% color for anchor text
%            citecolor=green,  % color for bibliographical citations
            citecolor=black,  % color for bibliographical citations
%            filecolor=magenta,% color for URLs which open local files
            filecolor=black,  % color for URLs which open local files
%            menucolor=red,    % color for Acrobat menu items
            menucolor=black,  % color for Acrobat menu items
%            pagecolor=red,    % color for links to other pages
            %pagecolor=black,  % color for links to other pages
%            urlcolor=cyan,    % color for linked URLs
            urlcolor=black,   % color for linked URLs
	          %bookmarks=true,         % create PDF bookmarks
	          bookmarksopen=false,    % don't expand bookmarks
	          bookmarksnumbered=true, % number bookmarks
	          pdftitle={model\_based\_methods\_for\_mts\_joao\_rodrigues},
            pdfauthor={João Rodrigues},
            pdfsubject={Model based methods for multivariate time series},
            pdfkeywords={Bayesian networks, Machine learning, Model-based methods, Multivariate time series, Boltzmann machines},
            pdfstartview=FitV,
            pdfdisplaydoctitle=true
            }

\usepackage{etoolbox}
\patchcmd{\thebibliography}
  {\settowidth}
  {\setlength{\parsep}{0pt}\setlength{\itemsep}{0pt plus 0.1pt}\settowidth}
  {}{}
\apptocmd{\thebibliography}
  {\scriptsize}
  {}{}

%%%%%%%%%%%%%%%%%%%%%%%%%%%%%%%%%%%%%%%%%%%%%%%%%%%%%%%%%%%%%%%%%%%%%%%%%%%%%%%%%%%%%%%%
% Title, authors and addresses

\title{\textbf{Determining the orientation of a RGB camera embedded on an artificial eye}}

\date{October 2019}
\author{Mariana Ribeiro dos Reis do Vale Martins \\ mariana.v.martins@tecnico.ulisboa.pt \\ \\ Instituto Superior T\'{e}cnico, Lisboa, Portugal}