%!TEX root = ../dissertation.tex

\begin{otherlanguage}{portuguese}
\begin{abstract}
\abstractPortuguesePageNumber
O olho tem seis músculos extra-oculares, precisando apenas de dois para orientar a fóvea para qualquer posição. Contudo, ainda não se entende como é que o cérebro determina que músculos utilizar. Construir um olho robótico pode ser valioso na identificação dos mecanismos por trás deste comportamento. Um modelo mecânico funcional de um olho biologicamente inspirado foi construído num projeto anterior. É indispensável neste protótipo determinar a orientação do olho com alta precisão, o que é atualmente feito com um \acrshort{imu}, que é significativamente impreciso devido ao desvio. Assim, nesta tese é estudada a adição de uma câmera \acrshort{rgb} para estimar a orientação 3D do olho, o que poderá melhorar a exactidão das medidas substancialmente. A posição da câmera no protótipo é tal que, ao girar o olho, há também uma translação associada ao movimento, que pode ser definida como uma função da rotação. Apesar da longa literatura em estimar a orientação usando uma câmera com pontos de interesse naturais, não há muita informação em relação a haver uma translação conhecida associada. Nesta tese, são comparados dois algoritmos do estado da arte e é apresentado um novo algoritmo que otimiza a estimativa usando essa particularidade. Os algoritmos são validados com um simulador e com o protótipo do olho, apresentando resultados promissores comparativamente com \acrshort{imu}.
% Keywords
\begin{flushleft}

\keywords{orientação da câmera, estimativa de rotação, geometria epipolar, procrustes, re-projeção de  pontos}

\end{flushleft}

\end{abstract}
\end{otherlanguage}