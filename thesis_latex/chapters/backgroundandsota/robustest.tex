%!TEX root = ../dissertation.tex

\section{Robust Estimation and Rejection of Image Sections}
\label{cha2:robustest}

Until now, the image points were assumed as given. However, it is necessary to use effective feature detection and matching algorithms to determine them. Wrongly matched pairs of points between the two images, or points with large location errors, can severely affect the precision of the rotation estimation. The reason for this is that all methods are least-square techniques that assume the noise which corrupts the data has zero mean. Hence, it is relevant to look into techniques of robust estimation.

In computer vision, it is very common to estimate the parameters of a model from image data. Robust estimation eliminates noise from the data. Points that don’t conform to a model are called outliers and are eliminated.

\subsubsection{\acrlong{ransac}}
\label{fipewnf}
One technique of robust estimation, is \acrfull{ransac}, first introduced by Martin A. Fischler and Robert C. Bolles in 1981 \cite{ransac}, where a small set of inliers is used to find a model and test all the 
other points against it. In this manner, it's possible to discover which points fit the model or not, and if they don’t consider them outliers. The final model is the one that has the most inliers. 

However, it is necessary to first define the said model. In the present study, the model could be obtained with the \acrlong{oppr} applied to the point matches. More specifically, 3 point matches would suffice to estimate the rotation between images that, applied to the remaining matches, could define which ones are outliers or inliers, leading to more matching accuracy.

Besides the accuracy, there is another interesting effect produced using this technique:

\begin{enumerate}
	\item Points that are closer to the camera are the ones that are more affected by the baseline/translation component, mentioned in section \ref{cha1:problemdef}. The ones further way are a closer fit to pure rotations.
	
	\item \acrshort{oppr} derives pure rotations as explained in section \ref{cha2:opprandsphere}. 
\end{enumerate}
Therefore by using \acrshort{ransac} and \acrshort{oppr} together, points matches closer to the camera, are naturally eliminated. This corresponds to image sections that would be more affected by translation and might prove beneficial to eliminate them. \cite{mono} \\

To summarize \acrshort{ransac} works as follows:

\begin{itemize}
	\item selects of a small random sample, "maybe inliers";
	\item fits a model to that sample;
	\item tests the model for the rest of the points matches. The ones that fit are "also inliers";
	\item checks if the total set of inliers ("maybe inliers" + "also inliers") are enough matches to have a good model;
	\item repeats all the steps for as many tries as desired in order to gather the largest set of inliers.
\end{itemize}

