% The definitions can be placed anywhere in the document body
% and their order is sorted by <symbol> automatically when
% calling makeindex in the makefile
%
% The \glossary command has the following syntax:
%
% \glossary{entry}
%
% The \nomenclature command has the following syntax:
%
% \nomenclature[<prefix>]{<symbol>}{<description>}
%
% where <prefix> is used for fine tuning the sort order,
% <symbol> is the symbol to be described, and <description> is
% the actual description.

% ----------------------------------------------------------------------
\nomenclature[A, 01]{$\bf m$}{Digital image point (in pixels)}
\nomenclature[A, 02]{$u, v$}{Digital image horizontal and vertical coordinates (in pixels)}
\nomenclature[A, 03]{$\bf m'$}{Image plane point (in meters)}
\nomenclature[A, 04]{$x', y'$}{Image plane horizontal and vertical coordinates (in meters)}
\nomenclature[A, 05]{$\bf m_d'$}{Image plane point (in meters) distorted}
\nomenclature[A, 06]{$x_d', y_d'$}{Image plane horizontal and vertical coordinates (in meters distorted)}
\nomenclature[A, 07]{$\bf M$}{3D point (in meters)}
\nomenclature[A, 08]{$X, Y, Z$}{3D point coordinates (in meters), Z is perpendicular to the image plane, X is horizontal and Y is vertical}
\nomenclature[A, 09]{$K$}{Intrinsics Matrix}
\nomenclature[A, 10]{$P$}{Projection Matrix}
\nomenclature[A, 11]{$\lambda$}{Depth}
\nomenclature[A, 12]{$f$}{Focal Length}
\nomenclature[A, 13]{$s$}{Skew}
\nomenclature[A, 14]{$c_x, c_y$}{Camera's principal point offset}
\nomenclature[A, 15]{$s_x, s_y$}{Pixel scalings}

\nomenclature[B, 01]{$E$}{Essential Matrix}
\nomenclature[B, 02]{$F$}{Fundamental Matrix}

\nomenclature[C, 01]{$R$}{Rotation Matrix}
\nomenclature[C, 02]{$\bf t$}{Translation vector}

