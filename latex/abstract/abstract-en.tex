%!TEX root = ../dissertation.tex

\begin{otherlanguage}{english}
\begin{abstract}
% Set the page style to show the page number
\thispagestyle{plain}
\abstractEnglishPageNumber

The brain is a highly complex system, even a simple task like looking at an object is still not fully understood. The eye has six extra-ocular muscles, but only needs two degrees of freedom to orient the fovea at any far position in the visual environment. How does the brain determine which muscles to contract? Bernstein theorized that it tries to find the optimal control solution, but which cost is minimized, energy, speed, accuracy? Building a robotic eye can prove valuable to identify the fundamental mechanisms behind these questions.

A working mechanical model of a biologically inspired eye has been built in a previous project. An indispensable aspect of this prototype is determining its orientation with high accuracy, which is currently done by an \acrshort{imu}. However, this method suffers from significant inaccuracy, due to drift. Hence, we investigated whether adding a \acrshort{rgb} camera to estimate the eye's 3D orientation could significantly improve accuracy and precision of the measurements.

The camera position on the prototype is such that when rotating the eye, there is also a translation associated to the movement, which is a fixed function of the rotation. Despite the long literature on orientation estimation using a camera with naturalistic visual features, there is not much literature concerning this constraint. We compare two adapted state-of-the art algorithms, and present a novel algorithm that optimally makes use of this constraint. We validated the algorithms with a simulator and on the real eye prototype, showing promising results when compared to the \acrshort{imu}.

% Keywords
\begin{flushleft}

\keywords{camera orientation, rotation estimation, epipolar geometry, procrustes, back projection}

\end{flushleft}

\end{abstract}
\end{otherlanguage}