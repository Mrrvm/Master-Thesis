%!TEX root = ../dissertation.tex

\section{Robust Estimation and Rejection of Image Sections}
\label{cha2:robustest}

In computer vision, it is very common to estimate the parameters of a model from image data. Robust estimation eliminates non-gaussian noise from the data. Points that don’t conform to a model are called outliers and are eliminated.

\subsubsection{\acrlong{ransac}}

One technique of robust estimation, is \acrfull{ransac}, first introduced by Martin A. Fischler and Robert C. Bolles in 1981 \cite{ransac}, where a small set of inliers is used to find a model and test all the 
other points against it. In this manner, it's possible to discover which points fit the model or not, and if they don’t consider them outliers. The final model is the one that has more inliers. 

However, it is necessary to have a way of defining the said model. In the case of this work, the model could be obtained through \acrlong{oppr} applied to the point matches. More specifically, 3 point matches would be enough to estimate the rotation between images that, applied to the remaining matches, could define which ones are outliers or inliers, leading to more matching accuracy.

Besides the accuracy, there is another interesting effect produced using this technique:

\begin{enumerate}
\item Points that are closer to the camera are the ones that are more affected by the baseline/translation component, mentioned in section \ref{cha1:problemdef}. The ones further way are a closer fit to pure rotations.

\item \acrshort{oppr} derives pure rotations as explained in section \ref{cha2:opprandsphere}. 
\end{enumerate}
Therefore by using \acrshort{ransac} and \acrshort{oppr} together, points matches closer to the camera, are naturally eliminated. This corresponds to image sections that would be more affected by translation and might prove beneficial to eliminate them. \cite{mono} 

To summarize \acrshort{ransac} works as following:

\begin{itemize}
	\item selection of a small random sample, "maybe inliers";
	\item fit of a model to that sample;
	\item test of the model for the rest of the points matches. The ones that fit are "also inliers";
	\item if "maybe inliers" + "also inliers" are enough points for a model to be considered good, then this inlier point matches are potential outcome;
	\item repeat all the steps for as many tries as desired in order to gather the biggest number of inliers.
\end{itemize}

