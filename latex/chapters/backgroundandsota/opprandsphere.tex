%!TEX root = ../dissertation.tex

\section{Orthogonal Procrustes Problem and Lack of Depth}
\label{cha2:opprandsphere}

After having established the camera model, it is now time to discuss some potential methods of determining image transformations, as needed for this project. 

The first of these methods is called the Orthogonal Procrustes Problem. Procrustes, a son of Poseidon in Greek mythology, made his victims fit in a wonderful all-fitting bed, either by stretching their limbs, or by cutting them off. Ultimately, he was fitted into his own bed by Theseus. Similar to the analysis held here, there are 3 elements to Procrustes's problem: $M_1$, the bed, $M_2$, the unlucky adventurer, and T, the fitting treatment. The solution to the problem is a matrix, T, that minimizes

\begin{equation}
\label{sec2:eq:procrustes}
\left \| M_1 - TM_2  \right \|^2 ,
\end{equation}
which transforms $M_2$ into $M_1$. 

Concerning the project's goal, $M_1$ and $M_2$ are clouds of 3D points observed by the camera obtained when looking at the same scene from two different points of view after a rotation, defined as $R$. Then the function to minimize is
\begin{equation}
\label{sec2:eq:ptrace}
\left \| M_1 - RM_2 \right \|^2 
\end{equation}
that through the Frobenius norm can be expanded as
\begin{equation}
\label{sec2:eq:ptrace}
\left \| M_1 - RM_2 \right \|^2 = \operatorname { trace } ( M_1^T M_1 + M_2^T M_2) - 2 \operatorname { trace } (M_2^T M_1 R),
\end{equation}
This means that minimizing (\ref{sec2:eq:procrustes}) with respect to $R$ is equivalent to maximizing the second term of (\ref{sec2:eq:ptrace}). By applying singular value decomposition (SVD) to $M_2^T M_1$, the latter term can be further simplified into

\begin{equation}
\label{sec2:eq:svd}
\operatorname { trace } (M_2^T M_1 R) = \operatorname { trace } ( U \Sigma V^T R) = \operatorname { trace } (\Sigma V^T R U) = \operatorname { trace } (\Sigma H )  = \sum _ { i = 1 } ^ { N } \sigma _ { i }h _ { i i } .
\end{equation}
The singular values of $\sigma _ { i }$ are all non-negative, and so the expression becomes maximum when $h _ { i i } = 1$ for $i=1,2,...,N$, since $H$, a product of orthogonal matrices, is an orthogonal matrix itself, thus having its maximal value when $H = I$. This results in $I = V^T R U$, and so $R=V U^T$.   

\subsection{Translation}
\label{sec2:trans}
Often, $M_1$ and $M_2$ may be associated with different origins. Hence, it is customary to consider  better fits of the configurations by allowing shape-preserving translations of the origin.
When translating $M_1$ and $M_2$ by $t_1$ and $t_2$, respectively, (\ref{sec2:eq:procrustes}) becomes
\begin{equation}
\label{sec2:eq:strans1}
\left \|(M_1 - \mathbf{1} \mathbf{t_1}^T) - R(M_2 - \mathbf{1} \mathbf{t_2}^T)\right \|^2,
\end{equation}
which can also be written as
\begin{equation}
\label{sec2:eq:strans2}
\left \| M_1  - RM_2 - \mathbf{1} \mathbf{t}^T\right \|^2,
\end{equation}
where $\mathbf{t}^T = \mathbf{t_1}^T R - \mathbf{t_2}^T$. The expression is minimized when $\mathbf{t}^T$ corresponds to the column-means of $M_1 R - M_2$, therefore a simple way of effecting this translation is to remove the column-means of $M_1$ and $M_2$, separately, before minimizing. This is called centering. The data then becomes expressed in terms of the mean deviations. \cite{procrustes}


