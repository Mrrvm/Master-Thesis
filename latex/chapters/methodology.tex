%!TEX root = ../dissertation.tex

\chapter{Methodology}
\label{methodology}

\section{Approach}

\subsection{Feature detection and matching}

\subsection{Filtering matches}

\subsection{Baseline derivation}

The transformation from the World to the Camera View 1 is
\begin{equation}
^W_{C_1}T = \begin{bmatrix}
I & -\mathbf{b}\\ 
\mathbf{0} & 1
\end{bmatrix}
\end{equation} and the transformation from the World to the Camera View 2 is
\begin{equation}
^W_{C_2}T = \begin{bmatrix}
R & -\mathbf{b}\\ 
\mathbf{0} & 1
\end{bmatrix}.
\end{equation}\\

Hence, the transformation from Camera View 1 to Camera View 2 is 
\begin{equation}
^{C_1}_{C_2}T = ^{W}_{C_2}T ^{C_1}_{W}T = ^{W}_{C_2}T ^{W}_{C_1}T^{-1} = 
\begin{bmatrix}
R & -\mathbf{b}\\ 
\mathbf{0} & 1
\end{bmatrix}
\begin{bmatrix}
I & \mathbf{b}\\ 
\mathbf{0} & 1
\end{bmatrix}
=
\begin{bmatrix}
R & R\mathbf{b}-\mathbf{b}\\ 
\mathbf{0} & 1
\end{bmatrix}.
\end{equation}
Thus, the translation is a function of the rotation and the baseline on the following way 
\begin{equation}
\mathbf{t}(R, \mathbf{b}) = R\mathbf{b}-\mathbf{b}.
\end{equation}


\subsection{\acrlong{grat}}

SAY THAT EPIPOLAR IS GOOD CUZ NO NEED TO DETERMINE Z, reference PAMI paper
IN TERMS OF R INSTEAD OF F
\subsection{\acrlong{mbpe}}
\label{MBaPE}
This method is a bundle adjustment of the previous one. It uses the rotation matrix obtained with OPPr and it tries to tune it to obtain a rotation and a translation dependent on the former, through 
\begin{align*}
	& \min_{R, Z_{e11}, ..., Z_{e1N}} \sum^N_{i=1} [(u_{e1i}-u_{1i})^2 + (u_{e2i}-u_{2i})^2 + (v_{e1i}-v_{1i})^2 + (v_{e2i}-v_{2i})^2]\\
	& \text{with} \ Z_{e1init} = \frac{1}{\sqrt{u_{1i}^2 + v_{1i}^2 + 1}} \ \text{and} \ R_{init} = R_{oppr},
\end{align*}
where $u_{1i}$ and $v_{1i}$ are the image points of the Camera View 1, $u_{2i}$ and $v_{2i}$ are the image points of the Camera View 2, $u_{e1i}$, $v_{e1i}$, $u_{e2i}$ and $v_{e2i}$ are the corresponding image points estimations and $Z_{e1i}$ is the depth of the Camera View 1.\\
The image point estimations are obtained the following way
\begin{align*}
	\mathbf{m_{e1}} = \frac{KR^T(Z_{e2}K^{-1}\mathbf{m_2}) - R^Tt)}{Z_{e1}}\\
	\mathbf{m_{e2}} = \frac{KR(Z_{e1}K^{-1}\mathbf{m_1}) + t)}{Z_{e2}}.
\end{align*} The depth is initialized by projecting the image points in a sphere.

\section{Implementation}

\subsection{Simulator}

\subsection{Real world}
